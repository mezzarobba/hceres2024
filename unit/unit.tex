% Define your personal commands and environments here.
%
% If you need additional packages, add them to the main TeX file
% (/hceres2024-main.tex), taking care not to break the rest of the
% document.

% Toggle instructions, page budgets, draft blocks (locally for this chapter)
% \settoggle{instructions}{false}

\begin{hceresinstructions}
  Depending on the area of evaluation, the unit bases its arguments on:

  \begin{itemize}
  \item
    The data provided in the "Characterization and Production Data"
    spreadsheet;
  \item
    The elements selected for the constitution of the portfolio;
  \item
    The presentation of its scientific themes in paragraph 3 of Chapter 1;
  \item
    The data provided in the appendix, if applicable.
  \end{itemize}

  For multi-team units, the four domains are first described at the unit
  level and then, for each team, the references considered relevant for
  the team are selected from among the domains. While it is neither
  appropriate nor necessary to deal with all the references for each team,
  those relating to production, attractiveness and inclusion in society
  should be given priority in this sequence.

  In the event that all the references should be addressed, care should be
  taken to respect the order of presentation.
\end{hceresinstructions}

\subsection*{Evaluation area 1. Profile, Resources and Organisation of the Unit}

\begin{hceresinstructions}
  This field is divided into three standards: adequacy of the research
  policy implemented by the unit to its human potential; financial and
  logistical means employed; responsible practices in matters of human
  resources, safety, and environment.
\end{hceresinstructions}

\subsubsection*{Standard 1. The unit has set itself relevant scientific
objectives.}

\begin{hceresinstructions}
  The unit expresses its vision of its research environment and its
  actors. It shows in particular how it takes into account the policy of
  its supervisory authorities in terms of research and development. It
  describes its scientific strategy and presents how it involves all its
  staff in the development of its research and valorisation policy.

  The unit analyses the scientific, economic, cultural and societal
  impacts of its policy and describes how it takes them into
  consideration.
\end{hceresinstructions}

\subsubsection*{Standard 2. The unit has resources adapted to its activity
profile and research environment and mobilizes them.}

\begin{hceresinstructions}
  The unit presents the financial resources at its disposal on a recurrent
  basis and the resources that it is able to mobilize, beyond the
  endowment allocated by its supervisory authorities. It describes its
  policy of pooling part of its resources to foster the emergence of
  innovative themes and to support collective research activities.

  The unit explains its policy regarding premises and scientific
  infrastructures or documentary resources. It shows how this is adapted
  to its scientific objectives.
\end{hceresinstructions}

\subsubsection*{Standard 3. The unit's functioning complies with the
rules and directives defined by its supervisors on human resources
management, safety, environment, ethical protocols and data as well as
scientific heritage protection.}

\begin{hceresinstructions}
  The unit defines its human resources policy. In particular, it describes
  how its human resources management respects gender equality and is
  non-discriminatory in terms of training, internal mobility and career
  development for its staff. It shows that it is mindful of the working
  conditions of its staff, their health and safety as well as the
  prevention of psycho-social risks. In particular, it sets out the
  measures taken to combat gender-based and sexual violence and
  discrimination.

  The unit describes all the procedures put in place to protect its
  scientific assets and computer systems.

  The unit indicates the measures it applies to prevent environmental
  risks resulting from its activity and to pursue sustainable development
  objectives. The unit specifies whether it has a sustainable development
  charter included in its internal regulations. In particular, it should
  show how it takes sustainable development criteria into account when
  defining research projects and experiments. It details its policy for
  managing staff missions and travel, and for managing waste, consumables
  and scrap. It describes the awareness-raising measures in place for
  students. It indicates how it evaluates its good practices in terms of
  environmental footprint.

  The unit describes its business continuity plan and how it anticipates
  emergency situations.
\end{hceresinstructions}

\subsubsection*{Synthetic self-evaluation}

\begin{hceresinstructions}
  The unit assesses its strengths and weaknesses against the standards for
  this area of evaluation.
\end{hceresinstructions}

\subsection*{Evaluation area 2. Attractiveness}

\begin{hceresinstructions}
  This field is divided into four standards which relate respectively to
  the scientific influence of the members of the unit, the quality of its
  supervision and reception policy in connection with the doctoral
  schools, its capacity to obtain funding (competitive calls for
  projects), and the quality of its equipment and its management.
\end{hceresinstructions}

\subsubsection*{Standard 1. The unit has an attractive scientific reputation and
contributes to the construction of the European research area.%
}

\begin{hceresinstructions}
  The unit explains the actions it is taking to develop its scientific
  influence. It illustrates its results in this area by the following
  highlights: invitations of unit members to conferences, organization of
  scientific events, editorial responsibilities, participation in research
  steering bodies, membership in institutions, prize winners, etc.
\end{hceresinstructions}

\subsubsection*{Standard 2. The unit is attractive for the quality of its staff
hosting policy.}

\begin{hceresinstructions}
  The unit presents its policy for welcoming new personnel. It mentions
  the modalities of reception and integration of both junior (PhD and
  post-doctorate) and senior (EC and C) researchers in the
  unit's research. It presents the results of this policy.
  It explains the support set up for research support staff.

  The unit emphasizes its capacity to host visiting researchers.

  The unit describes the implementation of the strategy of its supervisory
  authorities in terms of scientific integrity and open science.
\end{hceresinstructions}

\subsubsection*{Standard 3. The unit is attractive because of the recognition
gained through its success in competitive calls for projects.%
}

\begin{hceresinstructions}
  The unit describes its policy of responding to international, national
  and local calls for projects. It also presents the resulting results.

  It mentions how it finances doctoral and post-doctoral contracts,
  engineer and technician contracts, chairs and equipment from its own
  resources.

  The unit explains its involvement, at different levels, in schemes and
  projects financed by national investment programs (PIA, CPER, for
  example), and the benefits it derives from them.
\end{hceresinstructions}

\subsubsection*{Standard 4. The unit is attractive for the quality of its major
equipment and technological skills.}

\begin{hceresinstructions}
  The unit indicates all of its platforms, equipment and state-of-the-art
  demonstrators. It details its strategy for the development, maintenance
  and updating, as well as the opening to third parties, of its devices.
  It explains how it accesses the tools put in place by its supervisors to
  acquire and maintain heavy equipment.

  It describes and analyses the constitution of the technical and
  administrative team involved in the management of this equipment.
\end{hceresinstructions}

\subsubsection*{Synthetic self-evaluation}

\begin{hceresinstructions}
  The unit assesses its strengths and weaknesses against the standards for
  this evaluation area.
\end{hceresinstructions}

\subsection*{Evaluation area 3. Scientific production}

\begin{hceresinstructions}
  This field is divided into three references that deal respectively with
  the qualitative, quantitative and ethical aspects of scientific
  production and research results.
\end{hceresinstructions}

\subsubsection*{Standard 1. The scientific production of the unit meets quality
criteria.}

\begin{hceresinstructions}
  The unit analyses its scientific production. It relies in particular on
  the portfolio and the list of its production to show how this production
  is based on solid theoretical and methodological grounds, that it is
  original, that it presents a contribution to knowledge and that it
  reflects a national and international positioning of the research
  carried out by the unit.

  In this paragraph, the main scientific results of the unit will be taken
  from the paragraph ``Scientific subjects and their implications'' in the
  1st chapter of this document. These scientific highlights (discoveries,
  inventions, methodological advances, new concepts, breakthroughs, etc.),
  which are at the heart of the qualitative approach to evaluating the
  unit's research, will be detailed and may be the subject
  of a substantial development.
\end{hceresinstructions}

\subsubsection*{Standard 2. Scientific production is proportionate to the research
potential of the unit and shared out between its personnel.}

\begin{hceresinstructions}
  The unit presents its internal knowledge dissemination strategy. It
  analyses the possible imbalances in production between its different
  teams. It describes and analyses in particular the production of junior
  researchers. It mentions the measures taken to support the less active
  staff or to support doctoral and post-doctoral research staff in this
  respect. It emphasises the contribution of research support staff.
\end{hceresinstructions}

\subsubsection*{Standard 3. The scientific production of the unit respects the
principles of scientific integrity, ethics and open science. It complies
with the applicable guidelines in this field.}

\begin{hceresinstructions}
  The unit specifies the means implemented to guarantee the traceability
  and, where appropriate, the reproducibility of its results (laboratory
  notebooks, anti-plagiarism software, internal procedures for peer review
  - including rereading -, procedures for archiving data and source codes,
  etc.). It describes the means by which it assists its staff in the
  choice of appropriate media for dissemination (to avoid, for example,
  "predatory" conferences and journals) and for the fair consideration of
  contributions (in particular in co-signatures).

  The unit indicates the measures put in place to ensure that its
  scientific production is the result of research that respects human and
  animal life.

  The unit defines its open science policy.
\end{hceresinstructions}

\subsubsection*{Synthetic self-evaluation}

\begin{hceresinstructions}
  The unit assesses its strengths and weaknesses against the standards for
  this evaluation area.
\end{hceresinstructions}

\subsection*{Evaluation area 4. Contribution of Research Activities to Society}

\begin{hceresinstructions}
  In this field, the word "society" is meant in a broad sense. The
  integration of the research unit's activity in society
  may concern the economy, health, culture, the environment, etc. The
  field is broken down into three standards, which deal respectively with
  the unit's interactions with actors in the non-academic
  world, the products of its research for socio-economic and cultural
  actors and its interventions in the public sphere.
\end{hceresinstructions}

\subsubsection*{Standard 1. The unit stands out by the quality and quantity of its
non-academic interactions.}

\begin{hceresinstructions}
  The unit is asked to analyse its partnerships with actors in the
  cultural, economic and social world and to specify the modes of
  collaboration (conventions, contracts, etc.). It describes the extent of
  its activity with the non-academic world, for example through the
  pooling or hosting of personnel, the financing of doctoral students
  (CIFRE, thesis financed by contracts, etc.), the financing of its
  research activities, the organization of continuing education courses or
  participatory or collaborative science activities.

  The unit indicates how it takes on subjects of scientific,
  technological, social and cultural value, in coherence with its research
  policy. It emphasizes how its partnerships make it possible to meet
  environmental, societal or technological challenges.
\end{hceresinstructions}

\subsubsection*{Standard 2. The unit develops products for the cultural, economic and
social world.}

\begin{hceresinstructions}
  The unit presents its valorisation policy and the results obtained in
  terms of product development for the economic world (patents, licenses,
  support for company creation, expertise, participation in the drafting
  of standards, etc.).

  The unit describes its activity in disseminating its results to social,
  economic and cultural actors.
\end{hceresinstructions}

\subsubsection*{Standard 3. The team shares its knowledge with the general public and
takes part in debates in society.}

\begin{hceresinstructions}
  The unit explains and analyses its policy for sharing knowledge with the
  general public and in particular with school populations. It presents
  the measures taken to encourage its personnel to speak out in the public
  arena and to ensure that this is done with respect for scientific
  integrity and ethics.
\end{hceresinstructions}

\subsubsection*{Synthetic self-evaluation}

\begin{hceresinstructions}
  The unit assesses its strengths and weaknesses against the standards in
  this area of evaluation.
\end{hceresinstructions}
