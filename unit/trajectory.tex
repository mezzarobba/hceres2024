\section{Unit trajectory}

\begin{hceresinstructions}
  The trajectory is understood in two dimensions: the dynamics and
  ambition of research, on the one hand, and the organisation and life of
  the laboratory, on the other. It is described at the level of the unit
  and can then be broken down into teams.

  The unit is invited to describe, in a very synthetic manner, its
  long-term scientific history and to recall the objectives it had set for
  itself during the previous evaluation, the strategy it had put in place,
  and the challenges it intended to meet. These elements of scientific
  characterisation make it possible to carry out a critical analysis, to
  compare achievements with the initial objectives, to discuss successes
  and failures. The unit highlights the reorientations it has implemented.

  The unit specifies how it fits in today with its various activities
  (scientific, expertise, development, training, dissemination, etc.), at
  national and international levels, based on an analysis of the state of
  the art.

  The unit describes its scientific projection on the basis of its
  self-evaluation, its research achievements and the new research
  challenges identified. By placing itself in the perspective of its
  five-year scientific project, the unit presents its prospective vision
  of the evolution of its scientific field, its contribution to current
  issues and the positioning of the project in the national or
  international scientific field. It indicates its points of support, the
  points to be improved and the possibilities offered by its environment.
  It specifies the risks linked to this environment. It presents how it
  supports the emergence of new themes, risky research topics or rare
  disciplines.

  The unit sets out, in a forward-looking vision, its partnership strategy
  with the academic world (at local, national, European and international
  levels) and the socio-economic and cultural world. The unit is also
  invited to show how its project fits in with the strategy of its parent
  institutions and the strategy of the university site.

  The unit justifies the coherence of its research strategy with its
  resources and organisation: how its organisation and developments have
  served its scientific objectives and how its future organisation and
  requests for resources will meet its ambitions. In this paragraph, the
  unit specifies the number of staff, the resources to be mobilised and
  the structuring method (organisation, positioning and contribution of
  teams, synergies between teams, platforms) to support its orientations,
  scientific objectives and strategic choices. It presents an action plan
  on the new challenges facing laboratories: science and society, open
  science, environmental impact of the unit's activities,
  gender parity, scientific integrity, for example.
\end{hceresinstructions}
