% Define your personal commands and environments here.
%
% If you need additional packages, add them to the main TeX file
% (/hceres2024-main.tex), taking care not to break the rest of the
% document.

% Toggle instructions, page budgets, draft blocks (locally for this chapter)
% \settoggle{instructions}{false}

\begin{hceresinstructions}
  The unit drafts this document (SAD) by filling out the sections below
  and following the plan provided. The parts in green in the text provide
  guidance for the self-evaluation. They should be deleted once the
  document has been drafted.
\end{hceresinstructions}

\begin{extrainstructions}
  You may use this environment to add unit-specific instructions to your local
  template.
\end{extrainstructions}

\begin{draft}
  You may use this environment for draft content.
\end{draft}

\cite{DUMMY1}

\section{General information for the current contract}

\pagebudget{XXX pages}

\subsection{Unit Identification}

Unit name:

Acronym:

Label and number:

Main scientific field:

Scientific panels (in the Hcéres classification) by descending order of
importance:

\textbf{Panel 1}

\textbf{Panel 2}

\textbf{Panel 3}

\textbf{Panel 4}

\begin{hceresinstructions}
  Choose one of:
  \begin{itemize}
    \item SHS1 : Marchés et organisations
    \item SHS2 : Institutions, gouvernance et systèmes juridiques
    \item SHS3 : Le Monde social et sa diversité
    \item SHS4 : L'esprit humain et sa complexité
    \item SHS5 : Cultures et productions culturelles
    \item SHS6 : Histoire générale du passé et des savoirs
    \item SHS7 : Espace et relations homme/milieux
    \item ST1 : Mathématiques
    \item ST2 : Physique
    \item ST3 : Sciences de la Terre et de l'Univers
    \item ST4 : Chimie
    \item ST5 : Sciences pour l'ingénieur
    \item ST6 : Sciences et technologies de l'information et de la communication - STIC
    \item SVE1 : Biologie environnementale fondamentale et appliquée, évolution
    \item SVE2 : Productions végétales et animales (agronomie), biologie végétale et animale, biotechnologie et ingénierie des biosystèmes
    \item SVE3 : Molécules du vivant, biologie intégrative (des gènes et génomes aux systèmes), biologie cellulaire et du développement pour la science animale
    \item SVE4 : Immunité, infection et immunothérapie
    \item SVE5 : Neurosciences et troubles du système nerveux
    \item SVE6 : Physiologie et physiopathologie humaine, vieillissement
    \item SVE7 : Prévention, diagnostic et traitement des maladies humaines
  \end{itemize}
\end{hceresinstructions}

Executive team:

\begin{hceresinstructions}
  Detail the role of each member of the executive team.
\end{hceresinstructions}

List of the research unit's supervisory institutions and bodies:

Doctoral schools of affiliation:

\begin{hceresinstructions}
  Full name
\end{hceresinstructions}

\subsection{Presentation of the unit}

History, location of the unit:

Structure of the unit:

Teams, platforms, shared services, etc.:

Size and composition of the teams (if applicable) at 12/31/2023:

Scientific orientations of the unit and its teams (if applicable):

\begin{hceresinstructions}
  Name of the teams, present scientific orientation in item 3.
\end{hceresinstructions}

\subsection{Scientific subjects and their implications}

\begin{hceresinstructions}
  The research unit is invited to present its research themes over the reference
  period, placing them in an international context and on the basis of the key
  facts that it considers to be remarkable scientific advances. These scientific
  milestones will also be detailed in the response to reference 1 in evaluation
  area 3 of the self-assessment reference framework. They address scientific,
  technological, cultural, economic or societal issues. Where appropriate, this
  presentation of scientific themes may take into account the organisation of
  the research unit into teams which may present their own themes. This text may
  be substantial, but of reasonable length and adapted to the size of the
  research unit. It will be reflected in the activity profile defined in
  paragraph 4 of this chapter as well as in the portfolio, discussed in the
  second chapter of this self-evaluation document.

  The main elements of this paragraph will be briefly repeated, as a
  context, in the last chapter devoted to the research
  unit's trajectory.
\end{hceresinstructions}

\subsection{Activity profile}

\begin{hceresinstructions}
  The profile definition makes it possible to characterise, according to
  seven major categories (listed here in alphabetical order), all the
  activities carried out by the research collective. The activity profile
  is broken down at the level of the unit and, where appropriate, its
  teams.
\end{hceresinstructions}

\activityprofile[
  \begin{hceresinstructions}
    (please detail, one line maximum)
  \end{hceresinstructions}
]{0}{0}{0}{0}{0}{0}{0}

\subsection{Research environment}

\begin{hceresinstructions}
  The unit gives a synthetic presentation of the research and transfer
  structures in which it is involved, at the institution or site level:

  \begin{itemize}
  \item Contribution to a field of research (centre, institute, sector,
  district, campus, etc.) and description of it;

  \item involvement in a structure created by the Future Investments Programme
  (PIA), such as an IdEx, I-Site, LabEx, ÉquipEx, PEPR, EUR, IHU, etc.;

  \item Membership of research federations, platforms, an MSH, an OSU, etc;

  \item Involvement in regional clusters;

  \item Participation in development and transfer structures (incubators,
      SATT, IRT, ITE, etc.);

  \item Involvement in the continuum between research laboratories and care
  structures.

  \item ...
  \end{itemize}
\end{hceresinstructions}

\subsection{Consideration of the recommendations in the previous report%
}

\begin{hceresinstructions}
  The unit gives a synthetic presentation of the actions taken to
  implement the recommendations made in the previous evaluation of the
  unit and its teams. It evaluates its achievements.
\end{hceresinstructions}

\section{Portfolio introduction}


\begin{hceresinstructions}
  The portfolio supports the qualitative evaluation of the unit's
  activities. It is comprised of a set of items that the unit deems
  representative of its activities, missions and research environment.

  \textbf{The portfolio has an introduction explaining the choices made in
  its composition.} The introduction should reflect the research profile
  and items chosen for this portfolio. This introduction must not exceed
  an upper limit of 3 500 characters (spaces included) for single-team
  units and 7 000 characters for multi-team units. This introduction is
  the object of chapter 2.

  \textbf{The portfolio itself (the set of documents selected by the unit)
  will be compressed in a zip file containing the elements constituting
  it. This zip file will be submitted as an annex.}
  If the zip file
  exceeds 50 MB, the unit is invited to create a download link and to
  indicate it at the end of this chapter 2.

  The total number of items in the portfolio should take into account the
  size and structure of the research unit. It should also be kept within a
  reasonable limit so that the expert committee can take a thorough look
  at it. We propose the following framework as a guide:

  For a single-team unit, it amounts to:

  \begin{itemize}
  \item
    small unit (less than 19 permanent staff): five items including at
    least two publications;
  \item
    medium-sized unit (between 20 and 39 permanent staff): eight items,
    including at least four publications;
  \item
    Large units (40 staff or more): eleven items, including at least five
    publications;
  \end{itemize}

  For a multi-team unit, it amounts per team to:

  \begin{itemize}
  \item
    very large team (more than 20 permanent staff): maximum seven items,
    including at least three publications
  \item
    large team (between 10 and 19 permanent staff): maximum of five
    elements, including at least two publications
  \item
    medium-sized team (between 5 and 9 permanent staff): maximum of four
    elements, including two publications;
  \item
    small team (less than 4 permanent staff): maximum three items including one publication;
  \end{itemize}

  The unit may divide these elements between productions related to each
  team and productions at the unit level.

  For research units with more than 15 teams, the size of the portfolio
  will be discussed with the scientific advisor in charge of the unit.

  The portfolio may include the following elements:

  \begin{itemize}
  \item
    Productions representative of the unit's scientific
    positioning (knowledge front, theoretical positioning, methodological
    innovation, etc.) attesting in particular to its recognition at
    national, European and international level (e.g. articles, books,
    artistic creations);
  \item
    Elements highlighting the unit's involvement in
    supervision and training activities (initial or for the professional
    world) and testifying to the contributions of the
    unit's scientific activity to the specialisation of
    the institution's training offer (involvement in EUR
    projects, European universities or alliances for innovation, design of
    training courses for specific professional sectors, for example);
  \item
    Elements presenting social innovation dynamics (co-production of
    research with non-academic actors, research collaboration with citizen
    panels, for example);
  \item
    Elements illustrating valorisation and transfer actions (cooperation
    actions with local and regional authorities, actions in the field of
    public policy support, participation in technology scouting actions
    and other public-private partnerships, etc.) and contributions to
    socio-economic and cultural development (descriptive note on a
    significant R\&D contract, on the context of the creation of a
    start-up, for example);
  \item
    Elements highlighting research dissemination activities (organisation
    of events for the general public, production of audio-visual
    documents, podcasts, books, expert reports for social, economic,
    cultural and political players, etc.);
  \item
    Any other element that the unit deems relevant in order to appreciate
    the singular aspects of its activity profile.
  \end{itemize}
\end{hceresinstructions}
